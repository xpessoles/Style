%----------------------------------------------------------------------------------------
%	VARIOUS REQUIRED PACKAGES AND CONFIGURATIONS
%----------------------------------------------------------------------------------------

\usepackage[top=2cm,bottom=2cm,left=2cm,right=2cm,a4paper]{geometry} % Page margins

\usepackage{caption}
\usepackage{longtable}

\usepackage{graphicx} % Required for including pictures
\usepackage{float}
\usepackage[francais]{babel} % English language/hyphenation
\frenchbsetup{StandardLists=true} % Pour éviter la collision babel enumitem pour les listes

\usepackage{enumitem} % Customize lists
\setlist{nolistsep} % Reduce spacing between bullet points and numbered lists

\usepackage{booktabs} % Required for nicer horizontal rules in tables

\usepackage[svgnames]{xcolor} % Required for specifying colors by name
%\definecolor{ocre}{RGB}{243,102,25} % Define the orange color used for highlighting throughout the book
\definecolor{bleuxp}{RGB}{49,133,156} % Couleur ''bleue''
\definecolor{violetf}{RGB}{112,48,160} % Couleur ''violet''

\usepackage{enumitem}
\usepackage{pifont} % Pour les dinglist
\usepackage{multicol}
\usepackage{array} % Centrage vertical dans les tableaux
\usepackage{schemabloc}

% Requis par UPSTI pedagogique
\usepackage{etoolbox}         % Scripting
\usepackage{ifthen}           % Scripting
\usepackage{xargs}            % Pour arguments optionnels multiples
\usepackage{tabularx}         % Tableaux avancés

% Pour générer des QRCODE.
\usepackage{qrcode}

\usepackage{tikz} % Required for drawing custom shapes

% Définition des booleéns
\newif\iffiche
\newif\ifprof
\newif\iftd
\newif\ifcours
\newif\ifnormal
\newif\ifdifficile
\newif\iftdifficile
\newif\ifcolle
\newif\iflivret
\newif\ifcorrection % True si le corrigé existe, False Sinon

%----------------------------------------------------------------------------------------
%	FONTS
%----------------------------------------------------------------------------------------
\usepackage{bm}
\usepackage{siunitx}
\sisetup{output-decimal-marker = {,}}
\usepackage{textcomp}


\usepackage{avant} % Use the Avantgarde font for headings
%\usepackage{times} % Use the Times font for headings
%\usepackage{mathptmx} % Use the Adobe Times Roman as the default text font together with math symbols from the Sym­bol, Chancery and Com­puter Modern fonts
\usepackage[adobe-utopia]{mathdesign}
\usepackage{microtype} % Slightly tweak font spacing for aesthetics

%% Pour PdfLaTex
\usepackage[utf8]{inputenc} % Required for including letters with accents
\usepackage[T1]{fontenc} % Use 8-bit encoding that has 256 glyphs
% Pour xelatex
%\usepackage{fontspec}

\usepackage{amsmath}
%----------------------------------------------------------------------------------------
%	BIBLIOGRAPHY AND INDEX
%----------------------------------------------------------------------------------------

%\usepackage[style=alphabetic,citestyle=numeric,sorting=nyt,sortcites=true,autopunct=true,babel=hyphen,hyperref=true,abbreviate=false,backref=true,backend=biber]{biblatex}
\usepackage[style=alphabetic,citestyle=numeric,sorting=nyt,sortcites=true,autopunct=true,hyperref=true,abbreviate=false,backref=true,backend=biber]{biblatex}
\addbibresource{bibliography.bib} % BibTeX bibliography file
\defbibheading{bibempty}{}
\usepackage{calc} % For simpler calculation - used for spacing the index letter headings correctly
\usepackage{makeidx} % Required to make an index
\makeindex % Tells LaTeX to create the files required for indexing

%----------------------------------------------------------------------------------------
%	MAIN TABLE OF CONTENTS
%----------------------------------------------------------------------------------------

\usepackage{titletoc} % Required for manipulating the table of contents
\setcounter{tocdepth}{2}     % Dans la table des matieres
\setcounter{secnumdepth}{2}
\contentsmargin{0cm} % Removes the default margin


%----------------------------------------------------------------------------------------
%	PAGE HEADERS
%----------------------------------------------------------------------------------------

\usepackage{fancyhdr} % Required for header and footer configuration

%----------------------------------------------------------------------------------------
%	HYPERLINKS IN THE DOCUMENTS
%----------------------------------------------------------------------------------------

\usepackage{hyperref}
\hypersetup{hidelinks,backref=true,pagebackref=true,hyperindex=true,colorlinks=false,breaklinks=true,urlcolor= ocre,bookmarks=true,bookmarksopen=false,pdftitle={Title},pdfauthor={Author}}
\usepackage{bookmark}
\bookmarksetup{open,numbered,addtohook={%
\ifnum
	\bookmarkget{level}=0 % chapter
	\bookmarksetup{bold}%
\fi
\ifnum
	\bookmarkget{level}=-1 % part
	\bookmarksetup{color=bleuxp,bold}%
\fi}}

%\usepackage{pythontex}

\usepackage{listings}
\lstloadlanguages{Python}

\lstset{language=Python,
        columns=fullflexible,
%	  identifierstyle=\sffamily,
	  basicstyle=\ttfamily,
	  keywordstyle=\bf,
	  commentstyle=\normalfont,
	  numbers=none, numberstyle=\small, numbersep=30pt, %gestion numérotation
	  showstringspaces=false,%pour cacher le symbole d'espacement dans les chaînes
	  breaklines=true, %passage à la ligne automatique
	  prebreak=\mbox{$\swarrow$},
%	  includerangemarker=false, % pour inclure partiellement des fichiers
%	  rangeprefix=\#\ 
	  }

\newcommand{\lstinputpath}[1]{\lstset{inputpath=#1}}
%\usepackage{mdframed}



\usepackage{xcolor} % Required for specifying colors by name
%\definecolor{ocre}{RGB}{243,102,25} % Define the orange color used for highlighting throughout the book
 \definecolor{ocre}{RGB}{49,133,156} % Couleur ''bleue''
\definecolor{violetf}{RGB}{112,48,160} % Couleur ''violet''

\usepackage{wrapfig}

%% POLICE TEXTTT avec CESURE
\newcommand\textvtt[1]{{\normalfont\fontfamily{cmvtt}\selectfont #1}} 