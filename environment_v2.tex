\usepackage{subfig}
\usepackage{pgfplots}

% kaobox
\RequirePackage[most]{tcolorbox}

\definecolor{bleuxp}{RGB}{0,84,127}
\definecolor{bleuxpc}{RGB}{234,243,245}
\definecolor{orangexp}{RGB}{253,173,87}
\definecolor{orangexpc}{RGB}{254,230,204}
\definecolor{violetxp}{RGB}{128,0,128}
\definecolor{violetxpc}{RGB}{242,230,242}

\definecolor{rougexp}{RGB}{238,104,93}
\definecolor{rougexpc}{RGB}{247,184,179}





\setlength{\columnseprule}{0pt}

%%%%%%%%%%%
% Compteurs
\newcounter{cptApplication}[chapter]
\newcounter{cptTD}[chapter]
\newcounter{cptColle}[chapter]


% Pieds de page en couleur
\usepackage{eso-pic}
\AddToShipoutPictureBG{\color{bleuxpc}%
%\AtPageUpperLeft{\rule[-20mm]{\paperwidth}{20mm}}%
\AtPageLowerLeft{\rule{\paperwidth}{16mm}}}


\newtcolorbox{warn}[1][]{
	breakable,
	before skip=1.5\topskip,
	after skip=1.5\topskip,
	left skip=0pt,
	right skip=0pt,
	left=4pt,
	right=4pt,
	top=2pt,
	bottom=2pt,
	lefttitle=4pt,
	righttitle=4pt,
	toptitle=2pt,
	bottomtitle=2pt,
	sharp corners,
	boxrule=0pt,
	titlerule=.4pt,
	colback=RoyalBlue!25!White,
	colbacktitle=RoyalBlue!25!White,
	coltitle=black,
	colframe=black,
	fonttitle=\bfseries,
	#1
}


\newtcolorbox{exemple}[1][]{
	breakable,
	before skip=1.5\topskip,
	after skip=1.5\topskip,
	left skip=0pt,
	right skip=0pt,
	left=4pt,
	right=4pt,
	top=2pt,
	bottom=2pt,
	lefttitle=4pt,
	righttitle=4pt,
	toptitle=2pt,
	bottomtitle=2pt,
	sharp corners,
	boxrule=0pt,
	titlerule=.4pt,
	colback=RoyalBlue!25!White,
	colbacktitle=RoyalBlue!25!White,
	coltitle=black,
	colframe=black,
	fonttitle=\bfseries,
	title = Exemple -- #1
}



% Exemples en marge
\newtcolorbox{exemplem}[1][]{
	boxrule=0pt,
	colback=RoyalBlue!5!White,
	coltext= darkgray,
	sharp corners,enhanced,
	borderline west={2pt}{0pt}{RoyalBlue!25!White},
	left = 1pt,
	right = 1pt,
	top=1pt,
	bottom=1pt,
	colbacktitle=white,
	title = \textcolor{RoyalBlue!25!White}{\textit{Exemple}},
	%colframe=RoyalBlue,
	%detach title,
	%watermark text=\textit{\textsf{Exemple}}
}

\newtcolorbox{hypo}[1][]{
	breakable,
	before skip=1.5\topskip,
	after skip=1.5\topskip,
	left skip=0pt,
	right skip=0pt,
	left=4pt,
	right=4pt,
	top=2pt,
	bottom=2pt,
	lefttitle=4pt,
	righttitle=4pt,
	toptitle=2pt,
	bottomtitle=2pt,
	sharp corners,
	boxrule=0pt,
	titlerule=.4pt,
	colback=RoyalBlue!25!White,
	colbacktitle=RoyalBlue!25!White,
	coltitle=black,
	colframe=black,
	fonttitle=\bfseries,
	title = Hypothèse #1
}

\newtcolorbox{solution}[1][]{
	breakable,
	before skip=1.5\topskip,
	after skip=1.5\topskip,
	left skip=0pt,
	right skip=0pt,
	left=4pt,
	right=4pt,
	top=2pt,
	bottom=2pt,
	lefttitle=4pt,
	righttitle=4pt,
	toptitle=2pt,
	bottomtitle=2pt,
	sharp corners,
	boxrule=0pt,
	titlerule=.4pt,
	colback=violetxpc,
	colbacktitle=violetxpc,
	coltitle=black,
	colframe=violetxp,
	fonttitle=\bfseries,
	title = \'Eléments de correction #1
}

\newtcolorbox{corrige}[1][]{
	breakable,
	before skip=1.5\topskip,
	after skip=1.5\topskip,
	left skip=0pt,
	right skip=0pt,
	left=4pt,
	right=4pt,
	top=2pt,
	bottom=2pt,
	lefttitle=4pt,
	righttitle=4pt,
	toptitle=2pt,
	bottomtitle=2pt,
	sharp corners,
	boxrule=0pt,
	titlerule=.4pt,
	colback=violetxpc,
	colbacktitle=violetxpc,
	coltitle=black,
	colframe=violetxp,
	fonttitle=\bfseries,
	title = Correction #1
}

\newtcolorbox{defi}[1][]{
	breakable,
	before skip=1.5\topskip,
	after skip=1.5\topskip,
	left skip=0pt,
	right skip=0pt,
	left=4pt,
	right=4pt,
	top=2pt,
	bottom=2pt,
	lefttitle=4pt,
	righttitle=4pt,
	toptitle=2pt,
	bottomtitle=2pt,
	sharp corners,
	boxrule=0pt,
	titlerule=.4pt,
	colback=bleuxpc,
	colbacktitle=bleuxpc,
	coltitle=black,
	colframe=bleuxp,
	fonttitle=\bfseries,
	title = Définition -- #1
}



\newtcolorbox{demo}[1][]{
	breakable,
	before skip=1.5\topskip,
	after skip=1.5\topskip,
	left skip=0pt,
	right skip=0pt,
	left=4pt,
	right=4pt,
	top=2pt,
	bottom=2pt,
	lefttitle=4pt,
	righttitle=4pt,
	toptitle=2pt,
	bottomtitle=2pt,
	sharp corners,
	boxrule=0pt,
	titlerule=.4pt,
	colback=bleuxpc,
	colbacktitle=bleuxpc,
	coltitle=black,
	colframe=bleuxp,
	fonttitle=\bfseries,
	title = Démonstration #1
}

\newtcolorbox{obj}[1][]{
	breakable,
	before skip=1.5\topskip,
	after skip=1.5\topskip,
	left skip=0pt,
	right skip=0pt,
	left=4pt,
	right=4pt,
	top=2pt,
	bottom=2pt,
	lefttitle=4pt,
	righttitle=4pt,
	toptitle=2pt,
	bottomtitle=2pt,
	sharp corners,
	boxrule=0pt,
	titlerule=.4pt,
	colback = rougexpc!25!White,
	colbacktitle = rougexpc!25!White,
	coltitle=black,
	colframe=rougexp,
	fonttitle=\bfseries,
	title = Objectif#1
}

\newtcolorbox{methode}[1][]{
	breakable,
	before skip=1.5\topskip,
	after skip=1.5\topskip,
	left skip=0pt,
	right skip=0pt,
	left=4pt,
	right=4pt,
	top=2pt,
	bottom=2pt,
	lefttitle=4pt,
	righttitle=4pt,
	toptitle=2pt,
	bottomtitle=2pt,
	sharp corners,
	boxrule=0pt,
	titlerule=.4pt,
	colback=bleuxpc,
	colbacktitle=bleuxpc,
	coltitle=black,
	colframe=bleuxp,
	fonttitle=\bfseries,
	title = Méthode -- #1
}

\newtcolorbox{resultat}[1][]{
	breakable,
	before skip=1.5\topskip,
	after skip=1.5\topskip,
	left skip=0pt,
	right skip=0pt,
	left=4pt,
	right=4pt,
	top=2pt,
	bottom=2pt,
	lefttitle=4pt,
	righttitle=4pt,
	toptitle=2pt,
	bottomtitle=2pt,
	sharp corners,
	boxrule=0pt,
	titlerule=.4pt,
	colback=bleuxpc,
	colbacktitle=bleuxpc,
	coltitle=black,
	colframe=bleuxp,
	fonttitle=\bfseries,
	title = Résultat -- #1
}

\newtcolorbox{theoreme}[1][]{
	breakable,
	before skip=1.5\topskip,
	after skip=1.5\topskip,
	left skip=0pt,
	right skip=0pt,
	left=4pt,
	right=4pt,
	top=2pt,
	bottom=2pt,
	lefttitle=4pt,
	righttitle=4pt,
	toptitle=2pt,
	bottomtitle=2pt,
	sharp corners,
	boxrule=0pt,
	titlerule=.4pt,
	colback=bleuxpc,
	colbacktitle=bleuxpc,
	coltitle=black,
	colframe=bleuxp,
	fonttitle=\bfseries,
	title = Théorème -- #1
}

\newtcolorbox{prop}[1][]{
	breakable,
	before skip=1.5\topskip,
	after skip=1.5\topskip,
	left skip=0pt,
	right skip=0pt,
	left=4pt,
	right=4pt,
	top=2pt,
	bottom=2pt,
	lefttitle=4pt,
	righttitle=4pt,
	toptitle=2pt,
	bottomtitle=2pt,
	sharp corners,
	boxrule=0pt,
	titlerule=.4pt,
	colback=RoyalBlue!25!White,
	colbacktitle=RoyalBlue!25!White,
	coltitle=black,
	colframe=black,
	fonttitle=\bfseries,
	title = Propriété -- #1
}


\newtcolorbox{remarque}[1][]{
	breakable,
	before skip=1.5\topskip,
	after skip=1.5\topskip,
	left skip=0pt,
	right skip=0pt,
	left=4pt,
	right=4pt,
	top=2pt,
	bottom=2pt,
	lefttitle=4pt,
	righttitle=4pt,
	toptitle=2pt,
	bottomtitle=2pt,
	sharp corners,
	boxrule=0pt,
	titlerule=.4pt,
	colback=bleuxpc,
	colbacktitle=bleuxpc,
	coltitle=black,
	colframe=black,
	fonttitle=\bfseries,
	title = Remarque#1
}


\newtcolorbox{rem}[1][]{
	breakable,
	before skip=1.5\topskip,
	after skip=1.5\topskip,
	left skip=0pt,
	right skip=0pt,
	left=4pt,
	right=4pt,
	top=2pt,
	bottom=2pt,
	lefttitle=4pt,
	righttitle=4pt,
	toptitle=2pt,
	bottomtitle=2pt,
	sharp corners,
	boxrule=0pt,
	titlerule=.4pt,
	colback=bleuxpc,
	colbacktitle=bleuxpc,
	coltitle=black,
	colframe=black,
	fonttitle=\bfseries,
	title = Remarque#1
}

\newtcolorbox{remarques}[1][]{
	breakable,
	before skip=1.5\topskip,
	after skip=1.5\topskip,
	left skip=0pt,
	right skip=0pt,
	left=4pt,
	right=4pt,
	top=2pt,
	bottom=2pt,
	lefttitle=4pt,
	righttitle=4pt,
	toptitle=2pt,
	bottomtitle=2pt,
	sharp corners,
	boxrule=0pt,
	titlerule=.4pt,
	colback=bleuxpc,
	colbacktitle=bleuxpc,
	coltitle=black,
	colframe=black,
	fonttitle=\bfseries,
	title = Remarques#1
}

\newtcolorbox{test}[1][]{
	breakable,
	before skip=1.5\topskip,
	after skip=1.5\topskip,
	left skip=0pt,
	right skip=0pt,
	left=4pt,
	right=4pt,
	top=2pt,
	bottom=2pt,
	lefttitle=4pt,
	righttitle=4pt,
	toptitle=2pt,
	bottomtitle=2pt,
	sharp corners,
	boxrule=0pt,
	titlerule=.4pt,
	colback=bleuxpc,
	colbacktitle=bleuxpc,
	coltitle=black,
	colframe=black,
	fonttitle=\bfseries,
	title = Test #1
}

%% Pour le python %%
\usepackage{listingsutf8}

\lstset{language=Python,
  inputencoding=utf8/latin1,
  breaklines=true,
  basicstyle=\ttfamily\small,
  keywordstyle=\bfseries\color{green!40!black},
  commentstyle=\itshape\color{purple!40!black},
  identifierstyle=\color{blue},
  stringstyle=\color{orange},
  upquote = true,
  columns=fullflexible,
  backgroundcolor=\color{gray!10},frame=leftline,rulecolor=\color{gray}}  
  
\definecolor{mygreen}{rgb}{0,0.6,0}

\lstset{
     literate=%
         {é}{{\'e}}1    
         {è}{{\`e}}1    
         {ê}{{\^e}}1    
         {à}{{\`a}}1
		 {À}{{\`A}}1
         {â}{{\^a}}1	 
         {ô}{{\^o}}1    
         {ù}{{\`u}}1    
         {î}{{\^i}}1 
         {ç}{{\c{c}}}1
}

%%







%%%%%% Anthony
\usepackage{pgf}

% le dictionnaire
\pgfkeys{
  /stringcolor/.is family, /stringcolor,
  /stringcolor/.unknown/.code = {\pgfkeyssetvalue{/stringcolor/#1}{#1}},
  /stringcolor/GEO/.initial = black,
  /stringcolor/CIN/.initial = red,
  /stringcolor/STAT/.initial = green,
  /stringcolor/CHS/.initial = blue,
  /stringcolor/DYN/.initial = cyan,
  /stringcolor/TEC/.initial = magenta,
  /stringcolor/SLCI/.initial = brown,
  /stringcolor/PERF/.initial = darkgray,
  /stringcolor/COR/.initial = pink,
  /stringcolor/NL/.initial = orange,
  /stringcolor/SEQ/.initial = olive,
  /stringcolor/NUM/.initial = purple,
  /stringcolor/SYS/.initial = teal,
  /stringcolor/CT/.initial = violet,
}

% pour avoir la couleur
\newcommand{\colorForText}[1]{\pgfkeysvalueof{/stringcolor/#1}}
% la boîte
\newtcbox{\mybox}[1]{on line,
arc=0pt,outer arc=0pt,
colback=\colorForText{#1}!\pourCen!white,
colframe=\colorForText{#1}!50!black,
boxsep=0pt,left=1pt,right=1pt,top=2pt,bottom=2pt,
boxrule=0pt,bottomrule=1pt,toprule=1pt}

\newcommand{\xpComp}[2]{
%\mybox{#1} {\textbf{#1 -- #2}}}
\cpbox{#1}{#2} {\textbf{#1}}}

%%%% Fonctionnement : \xpComp{TEC}{00}
%%%%%% Anthony


% Pour la géométrie geo clair, geo geo foncé
%#1 : texte vertical
% Pour le corps, pas d'argument
\newtcbox{\cpbox}[2]{enhanced,nobeforeafter,tcbox raise base,boxrule=0.4pt,top=.8mm,bottom=0mm,
  right=0mm,left=4mm,arc=1pt,boxsep=2pt,before upper={\vphantom{dlg}},
  colframe= \colorForText{#1}!50!black,coltext=\colorForText{#1}!25!black,colback=\colorForText{#1}!10!white,
  fontupper=\sffamily\bfseries,
  overlay={\begin{tcbclipinterior}\fill[\colorForText{#1}!75!blue!50!white] (frame.south west)
    rectangle node[text=white,font=\sffamily\bfseries\footnotesize,rotate=90] {#2} ([xshift=4mm]frame.north west);\end{tcbclipinterior}}}


\newcommand{\progress}[1]{
    \begin{tikzpicture}
        \fill[bleuxpc, rounded corners =1 mm] (0,0) rectangle (#1/100*2.5,.4);
         \draw[bleuxp,thick, rounded corners =1 mm] (0,0) rectangle (2.5,.4);
         \node at (1.25,0.2) {\textcolor{black}{#1~\%}};
    \end{tikzpicture}
}

\newcommand{\CommaXp}{, }
\pgfkeys{
    /codecomp/.is family, /codecomp,
    /codecomp/.unknown/.code = {\pgfkeyssetvalue{/codecomp/#1}{#1}},
    /codecomp/GEO-00/.initial = Résoudre un problème de géométrie,
    /codecomp/GEO-01/.initial = Analyser la géométrie d'un mécanisme\CommaXp analyser des surfaces de contact\CommaXp réaliser des constructions géométriques,
    /codecomp/GEO-02/.initial = Modéliser un mécanisme en réalisant un schéma cinématique paramétré,
    /codecomp/GEO-03/.initial = Résoudre un problème de géométrie : déterminer la trajectoire d'un point ou déterminer une loi Entrée - Sortie,
    /codecomp/GEO-04/.initial = Évaluer expérimentalement une grandeur géométrique,
    /codecomp/CIN-00/.initial = Résoudre un problème de cinématique,
    /codecomp/CIN-01/.initial = Analyser un mécanisme\CommaXp réaliser un graphe de liaison,
    /codecomp/CIN-02/.initial = Déterminer un vecteur vitesse\CommaXp un torseur cinématique\CommaXp un vecteur accélération,
    /codecomp/CIN-03/.initial = Déterminer le rapport de transmission d'un transmetteur,
    /codecomp/CIN-04/.initial = Déterminer un loi ES cinématique\CommaXp utiliser l'hypothèse de RSG,
    /codecomp/CIN-05/.initial = Évaluer expérimentalement des grandeurs cinématiques,
    /codecomp/STAT-00/.initial = Résoudre un problème de statique,
    /codecomp/STAT-01/.initial = Analyser un problème en utilisant un graphe de structure,
    /codecomp/STAT-02/.initial = Modéliser les actions mécaniques locales\CommaXp globales\CommaXp frottement,
    /codecomp/STAT-03/.initial = Proposer une démarche de résolution en utilisant le PFS,
    /codecomp/STAT-04/.initial = Mettre en œuvre une démarche de résolution,
    /codecomp/STAT-05/.initial = Évaluer expérimentalement une action mécanique,
    /codecomp/CHS-00/.initial = Modéliser un mécanisme,
    /codecomp/CHS-01/.initial = Analyser un mécanisme en utilisant un graphe de liaisons,
    /codecomp/CHS-02/.initial = Simplifier un mécanisme en utilisant une liaison équivalente,
    /codecomp/CHS-03/.initial = Évaluer l'hyperstatisme d'un mécanisme,
    /codecomp/CHS-04/.initial = Simplifier un mécanisme pour le rendre isostatique,
    /codecomp/CHS-05/.initial = Analyser les conséquences de l'hyperstatisme d'un mécanisme,
    /codecomp/DYN-00/.initial = Résoudre un problème de dynamique,
    /codecomp/DYN-01/.initial = Analyser un problème\CommaXp définir une loi de mouvement,
    /codecomp/DYN-02/.initial = Analyser un mécanisme en utilisant un graphe de structure,
    /codecomp/DYN-03/.initial = Modéliser un solide et déterminer ses caractéristiques inertielles,
    /codecomp/DYN-04/.initial = Déterminer un torseur cinétique\CommaXp un torseur dynamique,
    /codecomp/DYN-05/.initial = Proposer une démarche de résolution en utilisant le PFD,
    /codecomp/DYN-06/.initial = Mettre en œuvre une démarche de résolution en utilisant le PFD,
    /codecomp/TEC-00/.initial = Résoudre un problème d'énergétique,
    /codecomp/TEC-01/.initial = Analyser un mécanisme en utilisant un graphe de structure,
    /codecomp/TEC-02/.initial = Déterminer les puissances intérieures,
    /codecomp/TEC-03/.initial = Déterminer les puissances extérieures,
    /codecomp/TEC-04/.initial = Déterminer l'inertie équivalente\CommaXp la masse équivalente\CommaXp l'énergie cinétique\CommaXp un travail,
    /codecomp/TEC-05/.initial = Proposer et mettre en œuvre une démarche de résolution,
    /codecomp/SLCI-00/.initial = Modéliser un SLCI,
    /codecomp/SLCI-01/.initial = Analyser un asservissement\CommaXp proposer une structure d'asservissement,
    /codecomp/SLCI-02/.initial = Modéliser un SLCI en utilisant la transformée de Laplace,
    /codecomp/SLCI-03/.initial = Modéliser un SLCI en utilisant un schéma-bloc,
    /codecomp/SLCI-04/.initial = Modéliser un SLCI en utilisant un modèle polyphysique,
	/codecomp/SLCI-05/.initial = Modéliser un SLCI à plusieurs entrées\CommaXp sous forme matricielle éventuellement,
    /codecomp/SLCI-06/.initial = Linéariser un comportement\CommaXp une équation\CommaXp simplifier un modèle,
    /codecomp/SLCI-07/.initial = Modéliser un système d'ordre 1 et d'ordre 2,
    /codecomp/SLCI-08/.initial = Déterminer une FTBO et une FTBF,
    /codecomp/SLCI-09/.initial = Identifier des fonctions de transfert (à partir d'un schéma-bloc)\CommaXp mettre sous forme canonique et identifier des constantes,
    /codecomp/SLCI-10/.initial = Déterminer et identifier une réponse temporelle,
    /codecomp/SLCI-11/.initial = Déterminer et identifier et analyser une réponse fréquentielle,
    /codecomp/PERF-00/.initial = Évaluer les performances d'un SLCI,
    /codecomp/PERF-01/.initial = Évaluer la stabilité en utilisant la BF\CommaXp les pôles de la BF,
    /codecomp/PERF-02/.initial = Évaluer la stabilité en utilisant les marges de la BO,
    /codecomp/PERF-03/.initial = Évaluer la rapidité de la réponse temporelle,
    /codecomp/PERF-04/.initial = Évaluer la rapidité à partir de la réponse fréquentielle de la BO,
    /codecomp/PERF-05/.initial = Évaluer la précision à partir du TVF,
    /codecomp/PERF-06/.initial = Évaluer la précision en utilisant la classe de la BO,
    /codecomp/COR-00/.initial = Corriger un SLCI,
    /codecomp/COR-01/.initial = Analyser un choix de correcteur (compensation de pôles\CommaXp nombre d'intégrations),
    /codecomp/COR-02/.initial = Régler un correcteur P graphiquement ou analytiquement,
    /codecomp/COR-03/.initial = Régler un correcteur PI graphiquement ou analytiquement,
    /codecomp/COR-04/.initial = Régler un correcteur à avance de phase,
    /codecomp/COR-05/.initial = Modéliser un correcteur numérique,
    /codecomp/COR-06/.initial = Implanter un correcteur sur une cible,
    /codecomp/NL-00/.initial = Modélisation des non linéarité d'un système,
    /codecomp/NL-01/.initial = Identifier une non linéarité,
    /codecomp/NL-02/.initial = Modéliser une non linéarité,
    /codecomp/SEQ-00/.initial = Modéliser un système combinatoire ou séquentiel,
    /codecomp/SEQ-01/.initial = Analyser un système séquentiel en utilisant un chronogramme\CommaXp analyser un système combinatoire en utilisant une table de vérité,
    /codecomp/SEQ-02/.initial = Modélisation par équation booléenne,
    /codecomp/SEQ-03/.initial = Modélisation par diagramme d'état,
    /codecomp/NUM-00/.initial = Résoudre un problème numériquement,
    /codecomp/NUM-01/.initial = Mettre un problème sous forme matricielle,
    /codecomp/NUM-02/.initial = Résolution de f(x)=0,
    /codecomp/NUM-03/.initial = Résolution d'une équation différentielle,
    /codecomp/NUM-04/.initial = Résoudre un problème numériquement,
    /codecomp/NUM-05/.initial = Résoudre un problème en utilisant l'apprentissage automatisé,
    /codecomp/SYS-00/.initial = Analyser et valider les performances d'un système,
    /codecomp/SYS-01/.initial = Réaliser une analyse structurelle\CommaXp flux\CommaXp effort,
    /codecomp/SYS-02/.initial = Analyser une solution technologique,
    /codecomp/SYS-03/.initial = Analyser un cahier des charges,
    /codecomp/SYS-04/.initial = Valider les performances d'un système vis-à-vis d'un cahier des charges,
    /codecomp/SYS-05/.initial = Analyser les résultats d'une simulation ou d'une expérimentation,
    /codecomp/SYS-06/.initial = Mesurer et analyser une grandeur physique,
}

\newcommand{\compForCodecomp}[2]{\xpComp{#1}{#2} \pgfkeysvalueof{/codecomp/#1-#2}}


% Description des compétences sur deux
\newcommand{\deuxcolxp}[2]{
\begin{minipage}[c]{1cm} \xpComp{#1}{#2} \end{minipage} \hspace{.3cm} %
\begin{minipage}[c]{8cm} \pgfkeysvalueof{/codecomp/#1-#2}\end{minipage} \\ \vspace{-0.2cm} \\}



\newcommand{\allSysComp}{%
\footnotesize
\xpComp{SYS}{00} \hspace{.25cm} \textbf{\pgfkeysvalueof{/codecomp/SYS-00}} \\ \vspace{-.1cm} \\
\indent \hspace{.5cm} \deuxcolxp{SYS}{01}
\indent \hspace{.5cm} \deuxcolxp{SYS}{02}
\indent \hspace{.5cm} \deuxcolxp{SYS}{03}
\indent \hspace{.5cm} \deuxcolxp{SYS}{04}
\indent \hspace{.5cm} \deuxcolxp{SYS}{05}
\indent \hspace{.5cm} \deuxcolxp{SYS}{06}
\normalsize
}

\newcommand{\allGeoComp}{% 
\footnotesize
\xpComp{GEO}{00} \hspace{.25cm} \textbf{\pgfkeysvalueof{/codecomp/GEO-00}} \\ \vspace{-.1cm} \\
\indent \hspace{.5cm} \deuxcolxp{GEO}{01}
\indent \hspace{.5cm} \deuxcolxp{GEO}{02}  
\indent \hspace{.5cm} \deuxcolxp{GEO}{03}  
\indent \hspace{.5cm} \deuxcolxp{GEO}{04} 
\normalsize
}

\newcommand{\allCinComp}{%
\footnotesize
\xpComp{CIN}{00} \hspace{.25cm} \textbf{\pgfkeysvalueof{/codecomp/CIN-00}} \\ \vspace{-.1cm} \\


\indent \hspace{.5cm} \deuxcolxp{CIN}{01}
\indent \hspace{.5cm} \deuxcolxp{CIN}{02}
\indent \hspace{.5cm} \deuxcolxp{CIN}{03}
\indent \hspace{.5cm} \deuxcolxp{CIN}{04}
\indent \hspace{.5cm} \deuxcolxp{CIN}{05}
\normalsize
}

\newcommand{\allStatComp}{%
\footnotesize
\xpComp{STAT}{00} \hspace{.25cm} \textbf{\pgfkeysvalueof{/codecomp/STAT-00}} \\ \vspace{-.1cm} \\
\indent \hspace{.5cm} \deuxcolxp{STAT}{01}
\indent \hspace{.5cm} \deuxcolxp{STAT}{02}
\indent \hspace{.5cm} \deuxcolxp{STAT}{03}
\indent \hspace{.5cm} \deuxcolxp{STAT}{04}
\indent \hspace{.5cm} \deuxcolxp{STAT}{05}
\normalsize
}

\newcommand{\allChsComp}{%
\footnotesize
\xpComp{CHS}{00} \textbf{\pgfkeysvalueof{/codecomp/CHS-00}} \\ \vspace{-.1cm} \\
\indent \hspace{.5cm} \deuxcolxp{CHS}{01}
\indent \hspace{.5cm} \deuxcolxp{CHS}{02}
\indent \hspace{.5cm} \deuxcolxp{CHS}{03}
\indent \hspace{.5cm} \deuxcolxp{CHS}{04}
\indent \hspace{.5cm} \deuxcolxp{CHS}{05}
\normalsize
}

\newcommand{\allDynComp}{%
\footnotesize
\xpComp{DYN}{00} \textbf{\pgfkeysvalueof{/codecomp/DYN-00}} \\ \vspace{-.1cm} \\
\indent \hspace{.5cm} \deuxcolxp{DYN}{01}
\indent \hspace{.5cm} \deuxcolxp{DYN}{02}
\indent \hspace{.5cm} \deuxcolxp{DYN}{03}
\indent \hspace{.5cm} \deuxcolxp{DYN}{04}
\indent \hspace{.5cm} \deuxcolxp{DYN}{05}
\indent \hspace{.5cm} \deuxcolxp{DYN}{06}
\normalsize
}

\newcommand{\allTecComp}{%
\footnotesize
\xpComp{TEC}{00} \textbf{\pgfkeysvalueof{/codecomp/TEC-00}} \\ \vspace{-.1cm} \\
\indent \hspace{.5cm} \deuxcolxp{TEC}{01}
\indent \hspace{.5cm} \deuxcolxp{TEC}{02}
\indent \hspace{.5cm} \deuxcolxp{TEC}{03}
\indent \hspace{.5cm} \deuxcolxp{TEC}{04}
\indent \hspace{.5cm} \deuxcolxp{TEC}{05}
\normalsize
}

\newcommand{\allSlciComp}{%
\footnotesize
\xpComp{SLCI}{00} \textbf{\pgfkeysvalueof{/codecomp/SLCI-00}} \\ \vspace{-.1cm} \\
\indent \hspace{.5cm} \deuxcolxp{SLCI}{01}
\indent \hspace{.5cm} \deuxcolxp{SLCI}{02}
\indent \hspace{.5cm} \deuxcolxp{SLCI}{03}
\indent \hspace{.5cm} \deuxcolxp{SLCI}{04}
\indent \hspace{.5cm} \deuxcolxp{SLCI}{05}
\indent \hspace{.5cm} \deuxcolxp{SLCI}{06}
\indent \hspace{.5cm} \deuxcolxp{SLCI}{07}
\indent \hspace{.5cm} \deuxcolxp{SLCI}{08}
\indent \hspace{.5cm} \deuxcolxp{SLCI}{09}
\indent \hspace{.5cm} \deuxcolxp{SLCI}{10}
\indent \hspace{.5cm} \deuxcolxp{SLCI}{11}
\normalsize
}

\newcommand{\allPerfComp}{%
\footnotesize
\xpComp{PERF}{00} \textbf{\pgfkeysvalueof{/codecomp/PERF-00}} \\ \vspace{-.1cm} \\
\indent \hspace{.5cm} \deuxcolxp{PERF}{01}
\indent \hspace{.5cm} \deuxcolxp{PERF}{02}
\indent \hspace{.5cm} \deuxcolxp{PERF}{03}
\indent \hspace{.5cm} \deuxcolxp{PERF}{04}
\indent \hspace{.5cm} \deuxcolxp{PERF}{05}
\indent \hspace{.5cm} \deuxcolxp{PERF}{06}
\normalsize
}

\newcommand{\allCorComp}{%
\footnotesize
\xpComp{COR}{00} \textbf{\pgfkeysvalueof{/codecomp/COR-00}} \\ \vspace{-.1cm} \\
\indent \hspace{.5cm} \deuxcolxp{COR}{01}
\indent \hspace{.5cm} \deuxcolxp{COR}{02}
\indent \hspace{.5cm} \deuxcolxp{COR}{03}
\indent \hspace{.5cm} \deuxcolxp{COR}{04}
\indent \hspace{.5cm} \deuxcolxp{COR}{05}
\indent \hspace{.5cm} \deuxcolxp{COR}{06}
\normalsize
}

\newcommand{\allNlComp}{%
\footnotesize
\xpComp{NL}{00} \textbf{\pgfkeysvalueof{/codecomp/NL-00}} \\ \vspace{-.1cm} \\
\indent \hspace{.5cm} \deuxcolxp{NL}{01} 
\indent \hspace{.5cm} \deuxcolxp{NL}{02} 
\normalsize
}

\newcommand{\allSeqComp}{%
\footnotesize
\xpComp{SEQ}{00} \textbf{\pgfkeysvalueof{/codecomp/SEQ-00}} \\ \vspace{-.1cm} \\
\indent \hspace{.5cm} \deuxcolxp{SEQ}{01} 
\indent \hspace{.5cm} \deuxcolxp{SEQ}{02} 
\indent \hspace{.5cm} \deuxcolxp{SEQ}{03} 
\normalsize
}

\newcommand{\allNumComp}{%
\footnotesize
\xpComp{NUM}{00} \textbf{\pgfkeysvalueof{/codecomp/NUM-00}} \\ \vspace{-.1cm} \\
\indent \hspace{.5cm} \deuxcolxp{NUM}{01} 
\indent \hspace{.5cm} \deuxcolxp{NUM}{02} 
\indent \hspace{.5cm} \deuxcolxp{NUM}{03} 
\indent \hspace{.5cm} \deuxcolxp{NUM}{04} 
\indent \hspace{.5cm} \deuxcolxp{NUM}{05} 
\normalsize
}


%\newcommand{\CommaXp}{, }
% Description des compétences sur une colonne... non sur 2... Comprenne qui voudra...
\newcommand{\uneColXp}[2]{
\begin{minipage}[c]{1cm} \xpComp{#1}{#2} \end{minipage} \hspace{.4cm} %
\begin{minipage}[c]{5.5cm} \pgfkeysvalueof{/codecomp/#1-#2}\end{minipage} \\ \vspace{.2cm} \\}


\newcommand{\allCompWide}{%
\textbf{\uneColXp{SYS}{00}}
\vspace{-.3cm}
\indent \hspace{.5cm} \uneColXp{SYS}{01}
\indent \hspace{.5cm} \uneColXp{SYS}{02}
\indent \hspace{.5cm} \uneColXp{SYS}{03}
\indent \hspace{.5cm} \uneColXp{SYS}{04}
\indent \hspace{.5cm} \uneColXp{SYS}{05}
\indent \hspace{.5cm} \uneColXp{SYS}{06}

\vspace{-.4cm}

\textbf{\uneColXp{GEO}{00}}
\vspace{-.3cm}
\indent \hspace{.5cm} \uneColXp{GEO}{01}
\indent \hspace{.5cm} \uneColXp{GEO}{02}  
\indent \hspace{.5cm} \uneColXp{GEO}{03}  
\indent \hspace{.5cm} \uneColXp{GEO}{04} 

\vspace{-.4cm}

\textbf{\uneColXp{CIN}{00}}
\vspace{-.3cm}
\indent \hspace{.5cm} \uneColXp{CIN}{01}
\indent \hspace{.5cm} \uneColXp{CIN}{02}
\indent \hspace{.5cm} \uneColXp{CIN}{03}
\indent \hspace{.5cm} \uneColXp{CIN}{04}
\indent \hspace{.5cm} \uneColXp{CIN}{05}

\vspace{-.4cm}

\textbf{\uneColXp{STAT}{00}}
\vspace{-.3cm}\indent \hspace{.5cm} \uneColXp{STAT}{01}
\indent \hspace{.5cm} \uneColXp{STAT}{02}
\indent \hspace{.5cm} \uneColXp{STAT}{03}
\indent \hspace{.5cm} \uneColXp{STAT}{04}
\indent \hspace{.5cm} \uneColXp{STAT}{05}

\vspace{-.4cm}

\textbf{\uneColXp{CHS}{00}}
\vspace{-.3cm}
\indent \hspace{.5cm} \uneColXp{CHS}{01}
\indent \hspace{.5cm} \uneColXp{CHS}{02}
\indent \hspace{.5cm} \uneColXp{CHS}{03}
\indent \hspace{.5cm} \uneColXp{CHS}{04}
\indent \hspace{.5cm} \uneColXp{CHS}{05}

\vspace{-.4cm}

\textbf{\uneColXp{DYN}{00}}
\vspace{-.4cm}
\indent \hspace{.5cm} \uneColXp{DYN}{01}
\indent \hspace{.5cm} \uneColXp{DYN}{02}
\indent \hspace{.5cm} \uneColXp{DYN}{03}
\indent \hspace{.5cm} \uneColXp{DYN}{04}
\indent \hspace{.5cm} \uneColXp{DYN}{05}
\indent \hspace{.5cm} \uneColXp{DYN}{06}

\vspace{-.4cm}

\textbf{\uneColXp{TEC}{00}}
\vspace{-.4cm}
\indent \hspace{.5cm} \uneColXp{TEC}{01}
\indent \hspace{.5cm} \uneColXp{TEC}{02}
\indent \hspace{.5cm} \uneColXp{TEC}{03}
\indent \hspace{.5cm} \uneColXp{TEC}{04}
\indent \hspace{.5cm} \uneColXp{TEC}{05}

\vspace{-.4cm}

\textbf{\uneColXp{SLCI}{00}}
\vspace{-.3cm}
\indent \hspace{.5cm} \uneColXp{SLCI}{01}
\indent \hspace{.5cm} \uneColXp{SLCI}{02}
\indent \hspace{.5cm} \uneColXp{SLCI}{03}
\indent \hspace{.5cm} \uneColXp{SLCI}{04}
\indent \hspace{.5cm} \uneColXp{SLCI}{05}
\indent \hspace{.5cm} \uneColXp{SLCI}{06}
\indent \hspace{.5cm} \uneColXp{SLCI}{07}
\indent \hspace{.5cm} \uneColXp{SLCI}{08}
\indent \hspace{.5cm} \uneColXp{SLCI}{09}
\indent \hspace{.5cm} \uneColXp{SLCI}{10}
\indent \hspace{.5cm} \uneColXp{SLCI}{11}

\vspace{-.4cm}

\textbf{\uneColXp{PERF}{00}}
\vspace{-.3cm}
\indent \hspace{.5cm} \uneColXp{PERF}{01}
\indent \hspace{.5cm} \uneColXp{PERF}{02}
\indent \hspace{.5cm} \uneColXp{PERF}{03}
\indent \hspace{.5cm} \uneColXp{PERF}{04}
\indent \hspace{.5cm} \uneColXp{PERF}{05}
\indent \hspace{.5cm} \uneColXp{PERF}{06}

\vspace{-.4cm}

\textbf{\uneColXp{COR}{00}}
\vspace{-.3cm}
\indent \hspace{.5cm} \uneColXp{COR}{01}
\indent \hspace{.5cm} \uneColXp{COR}{02}
\indent \hspace{.5cm} \uneColXp{COR}{03}
\indent \hspace{.5cm} \uneColXp{COR}{04}
\indent \hspace{.5cm} \uneColXp{COR}{05}
\indent \hspace{.5cm} \uneColXp{COR}{06}

\vspace{-.4cm}

\textbf{\uneColXp{NL}{00}}
\vspace{-.3cm}
\indent \hspace{.5cm} \uneColXp{NL}{01} 
\indent \hspace{.5cm} \uneColXp{NL}{02} 

\vspace{-.4cm}

\textbf{\uneColXp{SEQ}{00}}
\vspace{-.2cm}
\indent \hspace{.5cm} \uneColXp{SEQ}{01} 
\indent \hspace{.5cm} \uneColXp{SEQ}{02} 
\indent \hspace{.5cm} \uneColXp{SEQ}{03} 

\vspace{-.4cm}

\textbf{\uneColXp{NUM}{00}}
\vspace{-.3cm}
\indent \hspace{.5cm} \uneColXp{NUM}{01} 
\indent \hspace{.5cm} \uneColXp{NUM}{02} 
\indent \hspace{.5cm} \uneColXp{NUM}{03} 
\indent \hspace{.5cm} \uneColXp{NUM}{04} 
\indent \hspace{.5cm} \uneColXp{NUM}{05} 
}
